\documentclass{article}
\usepackage[utf8]{inputenc}
\usepackage{amsmath}
\usepackage{amssymb}

\title{Exercícios: Condições de Cauchy-Riemann}
\author{Não Trivial}
\date{}

\begin{document}

\maketitle

\section*{Problemas Práticos}

\subsection*{Exercício 1: Função Polinomial}
Considere \( f(z) = z^2 \).
\begin{enumerate}
    \item Escreva \( f(z) \) na forma \( u(x,y) + iv(x,y) \)
    \item Calcule \( \frac{\partial u}{\partial x} \), \( \frac{\partial u}{\partial y} \), \( \frac{\partial v}{\partial x} \), e \( \frac{\partial v}{\partial y} \)
    \item Verifique se as condições de Cauchy-Riemann são satisfeitas
    \item A função é analítica em todo o plano complexo? Justifique
\end{enumerate}

\subsection*{Exercício 2: Função com Conjugado}
Analise \( f(z) = \overline{z} \) (conjugado complexo):
\begin{enumerate}
    \item Expresse em termos de \( x \) e \( y \)
    \item Calcule as derivadas parciais necessárias
    \item Verifique as condições de Cauchy-Riemann
    \item Em quais pontos (se existirem) a função é diferenciável?
\end{enumerate}

\subsection*{Exercício 3: Função Mista}
Estude \( f(z) = z^2 + \overline{z} \):
\begin{enumerate}
    \item Separe em parte real e imaginária
    \item Aplique as condições de Cauchy-Riemann
    \item Determine as regiões onde a função é analítica
    \item Compare com o comportamento de \( z^2 \) isoladamente
\end{enumerate}

\section*{Casos Especiais}
\subsection*{Exercício 4: Exponencial Complexa}
Para \( f(z) = e^{\overline{z}} \):
\begin{enumerate}
    \item Expanda usando a definição de exponencial complexa
    \item Verifique as condições de Cauchy-Riemann
    \item Esta função é inteira? Justifique
\end{enumerate}

\section*{Critérios de Avaliação}
\begin{itemize}
    \item Separação correta em partes real/imag (2 pontos cada)
    \item Cálculo preciso das derivadas (3 pontos cada)
    \item Aplicação adequada das condições (3 pontos cada)
    \item Conclusão lógica sobre analiticidade (2 pontos cada)
\end{itemize}

\section*{Notas Pedagógicas}
\begin{itemize}
    \item As condições de Cauchy-Riemann são \textbf{necessárias mas não suficientes} para analiticidade
    \item A diferenciabilidade complexa exige também continuidade das derivadas
    \item Funções que envolvem \( \overline{z} \) geralmente falham nas condições
    \item A analiticidade tem implicações profundas (integral de Cauchy, séries de potências)
\end{itemize}

\end{document}