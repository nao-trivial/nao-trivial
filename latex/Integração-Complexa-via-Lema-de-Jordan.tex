\documentclass{article}
\usepackage{amsmath}
\usepackage{amssymb}
\usepackage{amsthm}

\begin{document}

\section*{Problema: Avaliação de \(\displaystyle \int_{-\infty}^{\infty} \frac{e^{i\lambda x}}{x^2 + a^2} \, dx\)}

\textbf{Dado:} \(\lambda \in \mathbb{R}\), \(a > 0\).

\subsection*{Solução via Análise Complexa}

\paragraph{Passo 1: Contorno no Plano Complexo}
Considere a função complexa:
\[
f(z) = \frac{e^{i\lambda z}}{z^2 + a^2}
\]
e o contorno \(C = [-R, R] \cup \Gamma_R\), onde \(\Gamma_R\) é um semicírculo de raio \(R\) no semiplano superior (para \(\lambda > 0\)) ou inferior (para \(\lambda < 0\)).

\paragraph{Passo 2: Singularidades}
As singularidades de \(f(z)\) ocorrem em:
\[
z = \pm ai
\]
Apenas \(z = ai\) está no semiplano superior se \(\lambda > 0\), e \(z = -ai\) no semiplano inferior se \(\lambda < 0\).

\paragraph{Passo 3: Aplicação do Lema de Jordan}
Para \(R \to \infty\), a integral sobre \(\Gamma_R\) tende a zero porque:
\[
\left| \int_{\Gamma_R} f(z) \, dz \right| \leq \frac{\pi}{|\lambda|} \cdot \frac{1}{R} \to 0 \quad \text{(Lema de Jordan)}.
\]

\paragraph{Passo 4: Teorema dos Resíduos}
A integral sobre \(C\) é dada por:
\[
\oint_C f(z) \, dz = 2\pi i \cdot \text{Res}(f, ai) \quad \text{(para \(\lambda > 0\))}.
\]

\paragraph{Passo 5: Cálculo do Resíduo}
O resíduo em \(z = ai\) é:
\[
\text{Res}(f, ai) = \lim_{z \to ai} (z - ai) \cdot \frac{e^{i\lambda z}}{(z - ai)(z + ai)} = \frac{e^{-\lambda a}}{2ai}.
\]

\paragraph{Passo 6: Resultado Final}
Para \(\lambda > 0\):
\[
\int_{-\infty}^{\infty} \frac{e^{i\lambda x}}{x^2 + a^2} \, dx = 2\pi i \cdot \frac{e^{-\lambda a}}{2ai} = \frac{\pi}{a} e^{-\lambda a}.
\]
Para \(\lambda < 0\), o resultado é análogo, resultando em:
\[
\int_{-\infty}^{\infty} \frac{e^{i\lambda x}}{x^2 + a^2} \, dx = \frac{\pi}{a} e^{-|\lambda| a}.
\]

\subsection*{Análise Assintótica}

A solução \(\displaystyle \frac{\pi}{a} e^{-|\lambda| a}\) revelar comportamentos distintos em regimes assintóticos:

\paragraph{Caso 1: \(|\lambda| a \gg 1\) (Decaimento Exponencial)}
Para valores grandes de \(|\lambda|\) ou \(a\), a exponencial domina:
\[
\int_{-\infty}^{\infty} \frac{e^{i\lambda x}}{x^2 + a^2} \, dx \sim \mathcal{O}\left(\frac{e^{-|\lambda| a}}{a}\right), \quad |\lambda| a \to \infty.
\]
Isso reflete o decaimento rápido da integral, consistente com o Lema de Jordan, que exige que \(f(z)\) se comporte bem no infinito.

\paragraph{Caso 2: \(|\lambda| a \ll 1\) (Regime Estático)}
Quando \(|\lambda| \to 0\) ou \(a \to 0\), a integral aproxima-se do caso estático:
\[
\int_{-\infty}^{\infty} \frac{1}{x^2 + a^2} \, dx = \frac{\pi}{a} \quad \text{(Resultado exato para \(\lambda = 0\))}.
\]
Neste limite, a exponencial \(e^{i\lambda x}\) torna-se irrelevante, e a integral recupera o valor clássico.

\paragraph{Interpretação Física}
A dependência em \(e^{-|\lambda| a}\) sugere que:
\begin{itemize}
    \item Para \(\lambda \neq 0\), a integral é suprimida exponencialmente pelo fator \(e^{-|\lambda| a}\).
    \item O parâmetro \(a\) atua como um "comprimento de amortecimento" para oscilações de frequência \(\lambda\).
\end{itemize}

\subsection*{Conclusão}
\[
\boxed{
\int_{-\infty}^{\infty} \frac{e^{i\lambda x}}{x^2 + a^2} \, dx = \frac{\pi}{a} e^{-|\lambda| a}
}
\]

\textbf{Observação:} A análise assintótica confirma que o resultado é consistente com os limites físicos e matemáticos esperados, validando a aplicação do Lema de Jordan e do Teorema dos Resíduos.

\end{document}