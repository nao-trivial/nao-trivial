\documentclass[a4paper,12pt]{book}
\usepackage[utf8]{inputenc}
\usepackage[T1]{fontenc}
\usepackage{amsmath, amssymb, amsthm}
\usepackage{hyperref}
\usepackage{graphicx}
\usepackage{lipsum} % Para texto de preenchimento

\title{Lógica Descomplicada: Como Pensar Melhor com a Matemática}
\author{Seu Nome Aqui}
\date{\today}

\begin{document}

% Capa e Apresentação
\maketitle
\newpage
\chapter*{Apresentação}
\addcontentsline{toc}{chapter}{Apresentação}
\lipsum[1-2] % Substitua por um texto introdutório pessoal

% Sumário
\tableofcontents
\newpage

% Introdução
\chapter*{Introdução}
\addcontentsline{toc}{chapter}{Introdução}
O que é lógica e por que ela importa?\newline
Como a lógica e a matemática se conectam à filosofia e à ciência?\newline
Exemplo prático: resolver um problema cotidiano usando lógica.

% Capítulos Principais
\chapter{Fundamentos da Lógica Formal}
\section{O que são proposições?}
\lipsum[3]
\section{Verdadeiros, falsos e tabelas-verdade}
\lipsum[4]
\section{Operadores lógicos (e, ou, não, implica)}
\lipsum[5]
\section{Exercício simples: construa sua primeira tabela-verdade}
\lipsum[6]

\chapter{Raciocínio Dedutivo e Indutivo}
\section{Diferença entre dedução e indução}
\lipsum[7]
\section{Exemplos históricos: Pitágoras e o raciocínio dedutivo}
\lipsum[8]
\section{Exercício: resolvendo problemas lógicos (com solução explicada)}
\lipsum[9]

\chapter{Aplicações Práticas do Raciocínio Lógico}
\section{Resolver dilemas cotidianos com lógica}
\lipsum[10]
\section{Como a lógica pode ajudar em decisões profissionais}
\lipsum[11]
\section{Desafio: crie um plano baseado em lógica para um problema fictício}
\lipsum[12]

\chapter{Filosofia e Matemática em Diálogo}
\section{Como filósofos usaram lógica para entender o mundo}
\lipsum[13]
\section{Introdução ao pensamento formal (dedução de teoremas simples)}
\lipsum[14]
\section{Reflexão: a importância das ideias no desenvolvimento humano}
\lipsum[15]

% Extras e Material Complementar
\chapter*{Extras e Material Complementar}
\addcontentsline{toc}{chapter}{Extras e Material Complementar}
Resumo visual com diagramas ou esquemas.\newline
QR codes com links para vídeos explicativos.\newline
Lista de referências e leituras recomendadas.

% Conclusão
\chapter*{Conclusão}
\addcontentsline{toc}{chapter}{Conclusão}
Convite para continuar aprendendo com seus outros materiais (futuros cursos ou mentorias).\newline
Mensagem inspiradora: "Desenvolver seu raciocínio lógico é um superpoder que transforma sua visão do mundo!"

\end{document}
